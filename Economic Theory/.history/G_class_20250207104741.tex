\documentclass[11pt]{elegantbook}
\usepackage{graphicx}
%\usepackage{float}
\definecolor{structurecolor}{RGB}{40,58,129}
\linespread{1.6}
\setlength{\footskip}{20pt}
\setlength{\parindent}{0pt}
\newcommand{\argmax}{\operatornamewithlimits{argmax}}
\newcommand{\argmin}{\operatornamewithlimits{argmin}}
\elegantnewtheorem{proof}{Proof}{}{Proof}
\elegantnewtheorem{claim}{Claim}{prostyle}{Claim}
\DeclareMathOperator{\col}{col}
\title{Ganesh Class}
\author{Wenxiao Yang}
\institute{Haas School of Business, University of California Berkeley}
\date{2025}
\setcounter{tocdepth}{2}
\extrainfo{All models are wrong, but some are useful.}

\cover{cover.png}

% modify the color in the middle of titlepage
\definecolor{customcolor}{RGB}{32,178,170}
\colorlet{coverlinecolor}{customcolor}
\usepackage{cprotect}


\bibliographystyle{apalike_three}

\begin{document}
\maketitle

\frontmatter
\tableofcontents

\mainmatter



\chapter{Digital/Information Addiction}
\section{Hyperbolic Discounting}
The classical discounting model discounts as $1,\delta,\delta^2,...$, i.e., $f(t)=\delta^t$ which discounts at a constant rate. However, the hyperbolic discounting model discounts faster in short run and slower in long run (i.e., we care more about now than future, and the difference of two future dates is small.)
\begin{equation}
    \begin{aligned}
        u_\textnormal{day1}\succ u_\textnormal{day2}\succ \cdots u_\textnormal{day365}\sim u_\textnormal{day366}
    \end{aligned}
    \nonumber
\end{equation}
The functional form can be given by
\begin{equation}
    \begin{aligned}
        f(t)=\left(1+\alpha t\right)^{-\frac{\gamma}{\alpha}}
    \end{aligned}
    \nonumber
\end{equation}
The discount rate is declining with time, $-\frac{f'(t)}{f(t)}=\frac{\gamma}{1+\alpha t}$.

\subsection{Quasi Hyperbolic Discounting}
\begin{equation}
    \begin{aligned}
        U_t=u_t+\beta\left[\delta u_{t+1}+\delta^2 u_{t+2}+\cdots+\right]
    \end{aligned}
    \nonumber
\end{equation}






\bibliography{ref}

\end{document}