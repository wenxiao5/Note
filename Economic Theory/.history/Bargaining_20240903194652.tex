\documentclass[11pt]{elegantbook}
\usepackage{graphicx}
%\usepackage{float}
\definecolor{structurecolor}{RGB}{40,58,129}
\linespread{1.6}
\setlength{\footskip}{20pt}
\setlength{\parindent}{0pt}
\newcommand{\argmax}{\operatornamewithlimits{argmax}}
\newcommand{\argmin}{\operatornamewithlimits{argmin}}
\elegantnewtheorem{proof}{Proof}{}{Proof}
\elegantnewtheorem{claim}{Claim}{prostyle}{Claim}
\DeclareMathOperator{\col}{col}
\title{Dynamic Programming and Bargaining}
\author{Wenxiao Yang}
\institute{Haas School of Business, University of California Berkeley}
\date{2024}
\setcounter{tocdepth}{2}
\extrainfo{All models are wrong, but some are useful.}

\cover{cover.png}

% modify the color in the middle of titlepage
\definecolor{customcolor}{RGB}{32,178,170}
\colorlet{coverlinecolor}{customcolor}
\usepackage{cprotect}


\bibliographystyle{apalike_three}

\begin{document}
\maketitle

\frontmatter
\tableofcontents

\mainmatter

\chapter{Dynamic Programming in Continuous Time}
This chapter is based on the notes of Jesús Fernández-Villaverde and Galo Nuño, Wikipedia, and the notes of Benjamin Moll.

\section{Optimal Control and Hamiltonian}
Consider a problem
\begin{equation}
    \begin{aligned}
        \max_{u(t)}\ &J=\int_{t_0}^{t_1}I(x(t),u(t),t)dt\\
        \textnormal{s.t. }&\dot{x}(t)=f(x(t),u(t),t)\\
    \end{aligned}
    \nonumber
\end{equation}
The solution method involves defining an ancillary function known as the control Hamiltonian
\begin{definition}[Hamiltonian]
    $\mathcal{H}(x(t),u(t),\lambda(t),t)\equiv I(x(t),u(t),t)+\lambda^T(t)f(x(t),u(t),t)$
\end{definition}

\begin{proposition}
    The \textbf{first-order necessary conditions} for a maximum are given by
    \begin{enumerate}
        \item Maximum principle:
        \begin{equation}
            \begin{aligned}
                \mathcal{H}_u(x(t),u(t),\lambda(t),t)=0
            \end{aligned}
            \nonumber
        \end{equation}
        \item State transition function:
        \begin{equation}
            \begin{aligned}
                \dot{x}(t)=\mathcal{H}_\lambda(x(t),u(t),\lambda(t),t)=f(x(t),u(t),t)
            \end{aligned}
            \nonumber
        \end{equation}
        \item Costate equations:
        \begin{equation}
            \begin{aligned}
                \dot{\lambda}(t)=-\mathcal{H}_x(x(t),u(t),\lambda(t),t)=-[I_x(x(t),u(t),t)+\lambda^T(t)f_x(x(t),u(t),t)]
            \end{aligned}
            \nonumber
        \end{equation}
        with boundary condition for co-state variable(s) $\lambda(t)$, called
        “transversality condition”
        \begin{equation}
            \begin{aligned}
                \lambda(t_1)=0 \textnormal{ or }\lim_{t_1 \rightarrow \infty}\lambda(t_1)=0 \textnormal{ for infinite time horizons}
            \end{aligned}
            \nonumber
        \end{equation}
    \end{enumerate}
    A \textbf{sufficient condition} for a maximum is the concavity of the Hamiltonian evaluated at the solution, i.e.
    \begin{equation}
        \begin{aligned}
            \mathcal{H}_{uu}\left(x^*(t),u^*(t),\lambda(t),t\right)\leq 0
        \end{aligned}
        \nonumber
    \end{equation}
    where $u^*(t)$ is the optimal control and $x^*(t)$ is the resulting optimal trajectory for the state variable. Therefore, the necessary conditions are sufficient if the functions $I(x(t),u(t),t)$ and $f(x(t),u(t),t)$ are concave in both $x(t)$ and $u(t)$.
\end{proposition}
\begin{proof}[Derivation from the Lagrangian]
    The Lagrangian is
    \begin{equation}
        \begin{aligned}
            \mathcal{L}=\int_{t_0}^{t_1}\left(I(x(t),u(t),t)+\lambda^T(t)\left[f(x(t),u(t),t)-\dot{x}(t)\right]\right)dt
        \end{aligned}
        \nonumber
    \end{equation}
    In order to eliminate $\dot{x}(t)$, we can use integration by parts on the last term:
    \begin{equation}
        \begin{aligned}
            \int_{t_0}^{t_1}\lambda^T(t)\dot{x}(t)dt&=\int_{t_0}^{t_1}\lambda^T(t)dx(t)\\
            &=\lambda^T(t)x(t)\big|_{t_0}^{t_1}-\int_{t_0}^{t_1}\dot{\lambda}^T(t)x(t)dt\\
            &=\lambda^T(t_1)x(t_1)-\lambda^T(t_0)x(t_0)-\int_{t_0}^{t_1}\dot{\lambda}^T(t)x(t)dt\\
        \end{aligned}
        \nonumber
    \end{equation}
    which can be substituted back into the Lagrangian expression to give
    \begin{equation}
        \begin{aligned}
            \mathcal{L}=\int_{t_0}^{t_1}\left(I(x(t),u(t),t)+\lambda^T(t)f(x(t),u(t),t)+\dot{\lambda}^T(t)x(t)\right)dt-\lambda^T(t_1)x(t_1)+\lambda^T(t_0)x(t_0)
        \end{aligned}
        \nonumber
    \end{equation}
    In the form of Lagrangian functional, the first-order necessary condition is, for $t\in(t_0,t_1)$
    \begin{equation}
        \begin{aligned}
            \underbrace{I_u(x(t),u(t),t)+\lambda^T(t)f_u(x(t),u(t),t)}_{=\mathcal{H}_u(x(t),u(t),\lambda(t),t)}=0\\
            \underbrace{I_x(x(t),u(t),t)+\lambda^T(t)f_x(x(t),u(t),t)}_{=\mathcal{H}_x(x(t),u(t),\lambda(t),t)}+\dot{\lambda}(t)=0
        \end{aligned}
        \nonumber
    \end{equation}
    If both the initial value $x(t_0)$ and terminal value $x(t_{1})$ are fixed, no conditions on $\lambda(t_{0})$ and $\lambda(t_{1})$ are needed. If the terminal value is free, as is often the case, the additional condition 
λ
(
t
1
)
=
0
{\displaystyle \mathbf {\lambda } (t_{1})=0} is necessary for optimality. The latter is called a transversality condition for a fixed horizon problem.
\end{proof}

\begin{definition}[Generic Deterministic Optimal Control Problem]
    The generic deterministic optimal control problem is
    \begin{equation}
        \begin{aligned}
            v(x_0)=\max_{\{\alpha(t)\}_{t\geq 0}}&\int_0^\infty e^{-\rho t}r(x(t),\alpha(t))dt\\
            \textnormal{s.t. }&\dot{x}(t)=f(x(t),\alpha(t))\\
            &x(0)=x_0
        \end{aligned}
        \nonumber
    \end{equation}
    \begin{note}
        $\frac{dx(t)}{dt}$ can be written as $\dot{x}(t)$.
    \end{note}
    Here, $x\in\mathbb{X}\subset \mathbb{R}^N$ is the \textbf{state} vector, $\alpha\in \mathbb{A}\subset \mathbb{R}^M$ is the \textbf{control} vector, $\rho>0$ is the \textbf{discount factor}, $f(\cdot):\mathbb{A}\times \mathbb{X} \rightarrow \mathbb{R}^N$ the \textbf{drift}, and $r(\cdot):\mathbb{A}\times \mathbb{X} \rightarrow \mathbb{R}$ is the instantaneous \textbf{reward} (utility).
\end{definition}
%By Lagrangian functional: $\mathcal{L}(x(t),\alpha(t),\lambda(t))=e^{-\rho t}r(x(t),\alpha(t))+\lambda (f(x(t),\alpha(t))-\dot{x}(t))$.
We can obtain conditions for an optimum using the Hamiltonian.
The current-value Hamiltonian is
    \begin{equation}
        \begin{aligned}
            \mathcal{H}(x,\alpha,\lambda)=r(x,\alpha)+\lambda f(x,\alpha)
        \end{aligned}
        \nonumber
    \end{equation}
    where $\lambda\in \mathbb{R}^N$ is the ``co-state'' vector.
\begin{note}
    $\frac{\partial\mathcal{H}(x,\alpha,\lambda)}{\partial x}$ and $\frac{\partial\mathcal{H}(x,\alpha,\lambda)}{\partial \alpha}$ can be written as $\mathcal{H}_x(x,\alpha,\lambda)$ and $\mathcal{H}_\alpha(x,\alpha,\lambda)$ respectively.
\end{note}
\begin{proposition}
    The first-order necessary conditions for an optimum are
    \begin{equation}
        \begin{aligned}
            &\mathcal{H}_\alpha(x(t),\alpha(t),\lambda(t))=0\\
            &\dot{\lambda}(t)=\rho\lambda(t)-\mathcal{H}_x(x(t),\alpha(t),\lambda(t))\\
            &\dot{x}(t)=f(x(t),\alpha(t))
        \end{aligned}
        \nonumber
    \end{equation}
    for all $t\geq 0$.
\end{proposition}



\chapter{\cite{fuchs2022dynamic}: Dynamic Bargaining with Private Information}

\section{``Classic'' Coase Conjecture}
Consider a seller facing a buyer has private value of the good $v\in[1,2]$ that is distributed according to an atomless distribution with full support, $F(v)$. The seller has cost $c\geq 0$ to serve the buyer. Every period of an infinite horizon game, the uninformed seller makes an offer $p_t$. If $p_t$ is accepted, the game ends with payoffs $v-p_t$ and $p_t-c$ for the buyer and the seller, respectively. If $p_t$ is rejected, the seller makes another offer at time $t+\Delta$. Both buyer and seller discount future with the rate $r$. Thus, $\Delta$ can be thought of as the commitment power of the seller.

Since higher-value buyers lose more from delay, the Skimming Property holds in any equilibrium.
\begin{lemma}[Skimming Property]
    If type $v$ is willing to accept an offer $p_t$, then all higher types strictly prefer to accept this offer.
\end{lemma}
Denote by $k_t$ the highest type remaining before the offer at time $t$ is made. Then, the seller beliefs at time $t$ are $v\in[1,k_t]$ with truncated distribution $F(v)/F(k_t)$, where $k_0=2$.


A distinction arises in this game: (1). $c<1$ (the gap case); (2). $c=1$ (the no-gap case).

As shown by \cite{fudenberg1985infinite} and \cite{gul1986foundations}, the game must end in a finite number of rounds (uniformly bounded for all $\Delta$). The intuition is: the difference between distinguishing $k_t$ and an immediate trading $1$ is $k_t-1$. If the $k_t-1$ is relative small to $1-c$. The seller prefers an immediate trade rather than keep bargaining. So, there is a critical $k_t$ that is close enough to $1$ and the seller offers $p_t=1$ to get an immediate trade.

By the backward induction, equilibrium must be unique (up to seller's randomization at time $0$). These equilibria have the property that the seller value depends only on the state of the game, $k_t$, a property called ``stationarity'' that is similar to the equilibria being Markov.

Stationary equilibria continue to exist when $c = 1$, but other nonstationary equilibrium can be constructed by \cite{ausubel1989reputation}.

Formally, the Coase conjecture (\cite{coase1972durability}) states that
\begin{proposition}[Coase Conjecture]
    As $\Delta \rightarrow 0$ (i.e., the seller loses all ability to commit to prices), prices fall to the lowest buyer valuation, $p_0 \rightarrow 1$, and there is no inefficient delay.
\end{proposition}


\section{Exogenously Interdependent Values}
Suppose the cost of selling to type $v$ is an increasing function of the buyer's type $c(v)$ with $c'(v)$. We assume that immediate trade is strictly efficient for all $v\in (1,2]$, $c(v)<v$.

The equilibrium dynamics depend on the degree of adverse selection.
\begin{enumerate}
    \item If there is little adverse selection, i.e., $\mathbb{E}[c(v)]\leq 1$, the continuous-time limit of the unique stationary equilibrium has immediate trade (\cite{deneckere2006bargaining}).
    \item If there is more adverse selection, $\mathbb{E}[c(v)]>1$, the seller would prefer not to trade at all rather than offering $p_0=1$ and trading with all types, so trade cannot be efficient in equilibrium.\\
    No trade cannot be an equilibrium either:
    \begin{enumerate}
        \item For no trade to be an equilibrium, prices would always have to be above $2$. If the price were ever lower, a mass of buyers $(2-\epsilon,2]$ would accept this price, contradicting to no trade.
        \item If the price were always above $2$, there would be a profitable deviation for the seller. The seller could offer $p_0 = c(2) < 2$ and then offer a sequence of prices that would lead to no losses.
    \end{enumerate}
    What then do equilibria look like?
\end{enumerate}
Suppose the continuous-time limit equilibrium strategy of the buyer is characterized by a continuous downward-sloping demand function, $P(v)$. Then, the seller's best response problem is to choose the speed to maximize her expected value given $k_t$:
\begin{equation}
    \begin{aligned}
        r V(k_t) = \max_{\dot{k}_t\in[0,\infty]}\left(P(k_t)-c(k_t)-V(k_t)\right)(-\dot{k}_t)\frac{f(k_t)}{F(k_t)}+V'(k_t)\dot{k}_t
    \end{aligned}
    \nonumber
\end{equation}
When trade happens, the seller collects $P(k_t)-c(k_t)$ and the game ends. Trade happens with a probability flow






















































































\bibliography{ref}




\end{document}