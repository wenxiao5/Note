\documentclass[11pt]{elegantbook}
\usepackage{graphicx}
%\usepackage{float}
\definecolor{structurecolor}{RGB}{40,58,129}
\linespread{1.6}
\setlength{\footskip}{20pt}
\setlength{\parindent}{0pt}
\newcommand{\argmax}{\operatornamewithlimits{argmax}}
\newcommand{\argmin}{\operatornamewithlimits{argmin}}
\elegantnewtheorem{proof}{Proof}{}{Proof}
\elegantnewtheorem{claim}{Claim}{prostyle}{Claim}
\DeclareMathOperator{\col}{col}
\title{Miguel Class}
\author{Wenxiao Yang}
\institute{Haas School of Business, University of California Berkeley}
\date{2024}
\setcounter{tocdepth}{2}
\extrainfo{All models are wrong, but some are useful.}

\cover{cover.png}

% modify the color in the middle of titlepage
\definecolor{customcolor}{RGB}{32,178,170}
\colorlet{coverlinecolor}{customcolor}
\usepackage{cprotect}


\bibliographystyle{apalike_three}

\begin{document}
\maketitle

\frontmatter
\tableofcontents

\mainmatter



\chapter{Pricing}
\section{Monopoly}
The firm decides its price $p$ to maximize $\Pi(p)=p\cdot D(p)-C(D(p))$, where $D(\cdot)$ is the demand function and $C(\cdot)$ is the cost function.

The monopoly problem is maximizing the profit
\begin{equation}
    \begin{aligned}
        \max_p \Pi(p)=p\cdot D(p)-C(D(p))
    \end{aligned}
    \nonumber
\end{equation}
The F.O.C. (first-order condition) is
\begin{equation}
    \begin{aligned}
        \frac{\partial \Pi(p)}{\partial p}=D(p)+pD'(p)-C'(D(p))D'(p)=0\\
    \end{aligned}
    \nonumber
\end{equation}
and the S.O.C. (second-order condition) is
\begin{equation}
    \begin{aligned}
        \frac{\partial \Pi^2(p)}{\partial p^2}<0
    \end{aligned}
    \nonumber
\end{equation}

The F.O.C. gives that
\begin{equation}
    \begin{aligned}
        (p-C')D'&=-D\\
        p-C'&=-\frac{D}{D'}\\
        \underbrace{\frac{p-C'}{p}}_\text{Lerner Index}&=-\frac{D}{D'p}\\
        &=-\frac{1}{\frac{d D}{d p}\frac{p}{D}}=-\frac{1}{\frac{\frac{d D}{D}}{\frac{d p}{p}}}:=\frac{1}{E}
    \end{aligned}
    \nonumber
\end{equation}
where $\frac{\frac{d D}{D}}{\frac{d p}{p}}<0$ is the elasticity of demand with respect to price. The absolute value of the elasticity is denoted by $E$.

$E$ is supposed to be greater than $1$, otherwise, the optimal price is negative.

In the demand function $D(p)=kp^{-E}$, where the (absolute) elasticity is constant. Its elasticity is $-Ekp^{-E-1}\frac{p}{kp^{-E}}=-E$.





\bibliography{ref}

\end{document}