\documentclass[11pt]{elegantbook}
\usepackage{graphicx}
%\usepackage{float}
\definecolor{structurecolor}{RGB}{40,58,129}
\linespread{1.6}
\setlength{\footskip}{20pt}
\setlength{\parindent}{0pt}
\newcommand{\argmax}{\operatornamewithlimits{argmax}}
\newcommand{\argmin}{\operatornamewithlimits{argmin}}
\elegantnewtheorem{proof}{Proof}{}{Proof}
\elegantnewtheorem{claim}{Claim}{prostyle}{Claim}
\DeclareMathOperator{\col}{col}
\title{Bargaining}
\author{Wenxiao Yang}
\institute{Haas School of Business, University of California Berkeley}
\date{2024}
\setcounter{tocdepth}{2}
\extrainfo{All models are wrong, but some are useful.}

\cover{cover.png}

% modify the color in the middle of titlepage
\definecolor{customcolor}{RGB}{32,178,170}
\colorlet{coverlinecolor}{customcolor}
\usepackage{cprotect}


\bibliographystyle{apalike_three}

\begin{document}
\maketitle

\frontmatter
\tableofcontents

\mainmatter

\chapter{\cite{fuchs2022dynamic}: Dynamic Bargaining with Private Information}

\section{``Classic'' Coase Conjecture}
Consider a seller facing a buyer has private value of the good $v\in[1,2]$ that is distributed according to an atomless distribution with full support, $F\$F(v)$.



\bibliography{ref}




\end{document}