\documentclass[11pt]{elegantbook}
\definecolor{structurecolor}{RGB}{40,58,129}
\linespread{1.6}
\setlength{\footskip}{20pt}
\setlength{\parindent}{0pt}
\newcommand{\argmax}{\operatornamewithlimits{argmax}}
\newcommand{\argmin}{\operatornamewithlimits{argmin}}
\elegantnewtheorem{proof}{Proof}{}{Proof}
\elegantnewtheorem{claim}{Claim}{prostyle}{Claim}
\DeclareMathOperator{\col}{col}
\title{\textbf{Game Theory}}
\author{Wenxiao Yang}
\institute{Haas School of Business, University of California Berkeley}
\date{2023}
\setcounter{tocdepth}{2}
\cover{cover.png}
\extrainfo{All models are wrong, but some are useful.}

% modify the color in the middle of titlepage
\definecolor{customcolor}{RGB}{9,119,119}
\colorlet{coverlinecolor}{customcolor}
\usepackage{cprotect}

\addbibresource[location=local]{reference.bib} % bib

\begin{document}

\maketitle
\frontmatter
\tableofcontents
\mainmatter



\chapter{Signalling Game}
Based on
\begin{enumerate}[$\circ$]
    \item "Kreps, D. M., \& Sobel, J. (1994). Signalling. \textit{Handbook of game theory with economic applications}, 2, 849-867."
    \item 
\end{enumerate}

\section{Canonical Game}
\begin{definition}[Canonical Game]
    \normalfont
    \begin{enumerate}
        \item There are two players: $\mathbf{S}$ (sender) and $\mathbf{R}$ (receiver).
        \item $\mathbf{S}$ holds more information than $\mathbf{R}$: the value of some random variable $t$ with support $\mathcal{T}$. (We say that $t$ is the \textbf{type} of $\mathbf{S}$)
        \item Prior belief of $\mathbf{R}$ concerning $t$ are given by a probability distribution $\rho$ over $\mathcal{T}$ (common knowledge)
        \item $\mathbf{S}$ sends a \textbf{signal $s\in \mathcal{S}$} to $\mathbf{R}$ drawn from a signal set $\mathcal{S}$.
        \item $\mathbf{R}$ receives this signal, and then takes an \textbf{action} $a\in \mathcal{A}$ drawn from a set $\mathcal{A}$ (which could depend on the signal $s$ that is sent).
        \item $\mathbf{S}$'s payoff is given by a function $u: \mathcal{T}\times \mathcal{S} \times \mathcal{A} \rightarrow \mathbb{R}$ and $\mathbf{R}$'s payoff is given by a function $v: \mathcal{T}\times \mathcal{S} \times \mathcal{A} \rightarrow \mathbb{R}$.
    \end{enumerate}
\end{definition}

\section{Nash Equilibrium}
\begin{definition}
    \normalfont
    A \textit{behavior strategy} for $\mathbf{S}$ is given by a function 
\end{definition}











\end{document}