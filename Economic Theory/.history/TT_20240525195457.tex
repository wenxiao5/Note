\documentclass[11pt]{elegantbook}
\definecolor{structurecolor}{RGB}{40,58,129}
\linespread{1.6}
\setlength{\footskip}{20pt}
\setlength{\parindent}{0pt}
\newcommand{\argmax}{\operatornamewithlimits{argmax}}
\newcommand{\argmin}{\operatornamewithlimits{argmin}}
\elegantnewtheorem{proof}{Proof}{}{Proof}
\elegantnewtheorem{claim}{Claim}{prostyle}{Claim}
\DeclareMathOperator{\col}{col}
\title{\textbf{Microeconomic Theory}}
\author{Wenxiao Yang}
\institute{Haas School of Business, University of California Berkeley}
\date{2024}
\setcounter{tocdepth}{2}
\cover{cover.png}
\extrainfo{All models are wrong, but some are useful.}

% modify the color in the middle of titlepage
\definecolor{customcolor}{RGB}{9,119,119}
\colorlet{coverlinecolor}{customcolor}
\usepackage{cprotect}

\addbibresource[location=local]{reference.bib} % bib

\begin{document}

\maketitle
\frontmatter
\mainmatter

\section{Equilibrium in Auctions with Entry}
\begin{enumerate}[$\circ$]
    \item Levin, D., \& Smith, J. L. (1994). Equilibrium in auctions with entry. The American Economic Review, 585-599.
\end{enumerate}
A single item offered to a group of $N$ potential bidders. There are two stage: each potential  entrant decides whether to enter by a fixed cost $c$, then an auction is conducted among $n$ participants (the number of bidders who enter).
\begin{assumption}
    \begin{enumerate}
        \item The seller and all potential bidders are risk-neutral.
        \item We assume the seller's valuation is $0$ and the possible value's range being $[\underline{v},\bar{v}]$, where $\underline{v}<0<\bar{v}$. This case is equivalent to the case that the seller's valuation is $-\underline{v}$ and the possible value's range being $[0,\bar{v}-\underline{v}]$.
        \item The domain of possible values for the item ($V$) and the domain of estimates ($x$) are compact: $V\in[0,\bar{v}-\underline{v}]$ and $x\in [0,\bar{x}]$.
        \item Information is symmetric and bidders randomly draw values from the same distribution.
        \item The auction mechanism ($m$) and the number of potential bidders ($N$) are common knowledge, and the number of actual bidders is revealed prior to stage 2.
        \item A unique symmetric Nash equilibrium exists and individual behavior conforms to the symmetric Nash equilibrium.
    \end{enumerate}
\end{assumption}

Given the number of bidders $n$, cost $c$, and mechanism $m$, the \textit{ex-ante} expected gain from entering and bidding according to the symmetric Nash equilibrium of each potential entrant is denoted by $\mathbb{E}[\pi\mid n,m]$.

If $\mathbb{E}[\pi\mid n,m]$ is decreasing in $n$, $\exists$ an unqiue integer $n^*$ such that $\mathbb{E}[\pi\mid n^*,m]\geq 0>\mathbb{E}[\pi\mid n^*+1,m]$. We focus on the case that $n^*\in (0,N)$.

A symmetric entry equilibrium must yield the same probability of entry for all potential bidders, which is denoted by $q^*\in (0,1)$ and each potential entrant must be indifferent between entering or not:
\begin{equation}
    \begin{aligned}
        \sum_{n=1}^N\left[
            \begin{pmatrix}
            N-1\\
            n-1
        \end{pmatrix}
        (q^*)^{n-1}(1-q^*)^{N-n}\mathbb{E}[\pi\mid n,m]
        \right]=0
    \end{aligned}
    \label{EAE_1}
\end{equation}
where $\begin{pmatrix}N-1\\n-1\end{pmatrix}(q^*)^{n-1}(1-q^*)^{N-n}$ is the probability that exactly $n-1$ rivals also enter. The number of bidders has mean $\bar{n}:=q^*N$ and variance $\bar{n}(1-q^*)$.

We focus on the mechanism that a bidder wins and pays for the item only if his bid is the highest.
\begin{note}
    Different to the original paper, we focus on the ``free entry'' case (i.e., the mechanism can be denoted by the reserve prices $R=\{R_1,...,R_N\}$, where $R_n$ means the reserve price enforced by the seller if $n$ bidders enter).
\end{note}
We let $T$ represent the event that trade occurs (i.e., the highest bid is greater than the reserve price) and let $T_n(R_n)$ represent the probability of trade given $n$ bidders enter and the seller's mechanism.

Using symmetry, a bidder's \textit{ex-ante} expected profit, conditional on entering an auction with $n$ bidders, can be written as $\frac{V_n-W_n}{n}-c$, where $V_n$ is the expected value of the item to the highest bidder and $W_n$ is the expected payment this bidder makes to the seller.

We use $\Omega$ to denote $\{R,c,N\}$ and $B_i(q,\Omega)$ to denote the $i^{th}$ bidder's expected profit from entering when all $N-1$ rivals are using arbitrary entry probability $q$:

\end{document}