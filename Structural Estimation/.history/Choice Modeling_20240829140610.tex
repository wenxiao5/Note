\documentclass[11pt]{elegantbook}
\usepackage{graphicx}
%\usepackage{float}
\definecolor{structurecolor}{RGB}{40,58,129}
\linespread{1.6}
\setlength{\footskip}{20pt}
\setlength{\parindent}{0pt}
\newcommand{\argmax}{\operatornamewithlimits{argmax}}
\newcommand{\argmin}{\operatornamewithlimits{argmin}}
\elegantnewtheorem{proof}{Proof}{}{Proof}
\elegantnewtheorem{claim}{Claim}{prostyle}{Claim}
\DeclareMathOperator{\col}{col}
\title{Structural Estimation}
\author{Wenxiao Yang}
\institute{Haas School of Business, University of California Berkeley}
\date{2024}
\setcounter{tocdepth}{2}
\extrainfo{All models are wrong, but some are useful.}

\cover{cover.png}

% modify the color in the middle of titlepage
\definecolor{customcolor}{RGB}{32,178,170}
\colorlet{coverlinecolor}{customcolor}
\usepackage{cprotect}


\bibliographystyle{apalike_three}

\begin{document}
\maketitle

\frontmatter
\tableofcontents

\mainmatter

\chapter{Homogenous Products}
How do we estimate demand for a homogenous product?
\section{\cite{working1927statistical}: OLS is not informative with endogeneity}
Demand and Supply for coffee:
\begin{equation}
    \begin{aligned}
        Q^d_t=\alpha_0+\alpha_1 P_t+U_t\\
        Q^s_t=\beta_0+\beta_1 P_t+V_t
    \end{aligned}
    \nonumber
\end{equation}
where $\alpha_1<0$ and $\beta_1>0$. Equilibrium price and quantity are given by
\begin{equation}
    \begin{aligned}
        Q^d_t=Q^s_t \Rightarrow \left\{\begin{matrix}
            P_t=\frac{\beta_0-\alpha_0}{\alpha_1-\beta_1}+\frac{V_t-U_t}{\alpha_1-\beta_1}\\
            Q_t=\frac{\alpha_1\beta_0-\alpha_0\beta_1}{\alpha_1-\beta_1}+\frac{\alpha_1V_t-\beta_1U_t}{\alpha_1-\beta_1}
        \end{matrix}\right.
    \end{aligned}
    \nonumber
\end{equation}
\begin{note}
    Price is a function of both error terms, and we can't use a clever substitution to cancel things out.
\end{note}
We can find that price is positively correlated with demand shift $U_t$ and negatively correlated with supply shift $V_t$.

Consider the OLS estimator: $\hat{\alpha}_1=\hat{\beta}_1=\frac{\textnormal{Cov}(P_t,Q_t)}{\textnormal{Var}(P_t)}$. Since we have
\begin{equation}
    \begin{aligned}
        \left.\begin{matrix}
            \textnormal{Cov}(P_t,Q_t)=\alpha_1\textnormal{Var}(P_t)+\textnormal{Cov}(P_t,U_t)\\
            \textnormal{Cov}(P_t,Q_t)=\beta_1\textnormal{Var}(P_t)+\textnormal{Cov}(P_t,V_t)
        \end{matrix}\right\} \Rightarrow \left\{\begin{matrix}
            \textnormal{Bias}(\alpha_1)=|\hat{\alpha}_1-\alpha_1|=\frac{\textnormal{Cov}(P_t,U_t)}{\textnormal{Var}(P_t)}\\
            \textnormal{Bias}(\beta_1)=|\hat{\beta}_1-\beta_1|=\frac{\textnormal{Cov}(P_t,V_t)}{\textnormal{Var}(P_t)}\\
        \end{matrix}\right.
    \end{aligned}
    \nonumber
\end{equation}
When $\textnormal{Cov}(U_t,V_t)=0$, the OLS estimator can be written as
\begin{equation}
    \begin{aligned}
        \hat{\alpha}_1=\hat{\beta}_1=\frac{\alpha_1\textnormal{Var}(V_t)+\beta_1\textnormal{Var}(U_t)}{\textnormal{Var}(V_t)+\textnormal{Var}(U_t)}
    \end{aligned}
    \nonumber
\end{equation}
More variation in supply $V_t$ gives a better estimate of demand and more variation in demand $U_t$ gives a better estimate of supply.

The OLS is not informative about the economic demand function (or supply function).

To deal with this problem, we need an excluded instrument that shifts one curve without affecting the other. Then, we can use this to form a 2SLS estimate.


\section{\cite{angrist2000interpretation}}
\subsection{Motivation}
What if we don't generality know which kind of heterogeneity we face?

There are four cases that are ranked in increasing complexity:
\begin{enumerate}
    \item Linear system with constant coefficients:
    \begin{equation}
        \begin{aligned}
            q^d_t(p,z,x)=\alpha_0+\alpha_1 p+\alpha_2 z+\alpha_3 x+\epsilon_t\\
            q^s_t(p,z,x)=\beta_0+\beta_1 p+\beta_2 z+\beta_3 x+\eta_t
        \end{aligned}
        \nonumber
    \end{equation}
    \item Linear system with non-constant coefficients:
    \begin{equation}
        \begin{aligned}
            q^d_t(p,z,x)=\alpha_{0t}+\alpha_{1t} p+\alpha_{2t} z+\alpha_{3t} x +\epsilon_t\\
            q^s_t(p,z,x)=\beta_{0t}+\beta_{1t} p+\beta_{2t} z+\beta_{3t} x +\eta_t
        \end{aligned}
        \nonumber
    \end{equation}
    \item Nonlinear system with constant shape (separable):
    \begin{equation}
        \begin{aligned}
            q^d_t(p,z,x)=q^d(p,z,x)+\epsilon_t\\
            q^s_t(p,z,x)=q^s(p,z,x)+\eta_t
        \end{aligned}
        \nonumber
    \end{equation}
    \item Nonlinear system with time-varying shape (non-separable): any forms of $q^d_t(p,z,x)$ and $q^s_t(p,z,x)$.
\end{enumerate}

\subsection{Model}
We assume the regularity conditions (existence of first and second moment and being stationary) and $q^d_t(p,z,x)$, $q^s_t(p,z,x)$ are continuously differentiable in $p$.

\paragraph*{Instrumental Variable}
Assume binary instrument $z_t\in\{0,1\}$ to make things easier. And $z_t\in\{0,1\}$ is a valid instrument in $q^d_t$, i.e., it satisfies
\begin{enumerate}
    \item Exclusion: $q^d_t(p_t,z=1,x_t)=q^d_t(p_t,z=0,x_t)=q^d_t(p_t,x_t)$.
    \item Relevance: $q^s_t(p_t,z=1,x_t)\neq q^s_t(p_t,z=0,x_t)$ for some period $t$.
    \item Independence: $\epsilon_t,\eta_t,z_t$ are mutually independent conditional on $x_t$.
\end{enumerate}

Suppose $z=1$ denote ``stormy at sea'' and $z=0$ denote ``calm at sea''. (Offshore weather makes fishing more difficult but doesn't change onshore demand.)

2SLS can work in linear models,
\begin{equation}
    \begin{aligned}
        \hat{\alpha}_{1,0}=\frac{\widehat{\mathbb{E}[q_t|z_t=1]}-\widehat{\mathbb{E}[q_t|z_t=0]}}{\widehat{\mathbb{E}[p_t|z_t=1]}-\widehat{\mathbb{E}[p_t|z_t=0]}}\stackrel{P}{\longrightarrow}\frac{\mathbb{E}[q_t|z_t=1]-\mathbb{E}[q_t|z_t=0]}{\mathbb{E}[p_t|z_t=1]-\mathbb{E}[p_t|z_t=0]}\equiv \alpha_{1,0}
    \end{aligned}
    \nonumber
\end{equation}
but it is not an estimator of a structural parameter in nonliner models.

\begin{claim}
    Authors make the point that IV estimator identifies something about relationship between $p$ and $q$, without identifying deep structural parameters.
\end{claim}

Some assumptions are needed to interpret the IV estimator.
\begin{assumption}
    \begin{enumerate}
        \item Observed price is market clearing price $q^d_t(p_t)=q^s_t(p_t,z_t)$ for all $t$. (i.e., no friction).
        \item For each value of $z$ and $t$, there is a unique market clearing price, $\tilde{p}(z,t)$, such that
        \begin{equation}
            \begin{aligned}
                q^d_t(\tilde{p}(z,t))=q^s_t(\tilde{p}(z,t),z)
            \end{aligned}
            \nonumber
        \end{equation}
        $\tilde{p}(z,t)$ is the potential price under any counterfactual $(z,t)$.
    \end{enumerate}
\end{assumption}


\chapter{Random Utility Models}
\section{Random Utility Models}
Let $j=1,...,J_i$ index the ``inside goods'' available to consumer $i$ while $j=0$ denotes the outside good. A consumer's choice set is characterized by $J_i$ and a set $\chi_i$, which may include
\begin{enumerate}[$\circ$]
    \item observed characteristics of consumer $i$,
    \item observed characteristics of goods (including prices),
    \item observed characteristics of the local market,
    \item and characteristics of the market or goods that are unobserved to the researcher.
\end{enumerate}
Each consumer $i$ has a (conditional indirect) utility $u_{ij}$ for good $j$. Consumer knows her utilities for all goods and chooses the good with the highest utility.

In this model, the heterogeneity of consumer preferences is modeled by random utilities:
\begin{definition}[Random Utility Model]
    Given the choice set $(J_i, \chi_i)$, each consumer's utility vector $(u_{ij})_{j=0,1,...,J_i}$ is an independent draw from a joint distribution $F_u(\cdot\mid J_i, \chi_i)$.
\end{definition}
Since only the ordinal ranking of goods matters for a consumer's behavior, we can normalize the location and scale of each consumer's utility vector without loss of generality. We assume that ``ties'' ($u_{ij}=u_{ik}$ for some $j\neq k$) occur with probability zero in the distribution $F_u(\cdot\mid J_i, \chi_i)$. Then, we can represent consumer $i$'s choice with the vector $(q_{i1},...,q_{iJ_i})$, where
\begin{equation}
    \begin{aligned}
        q_{ij}=\mathbf{1}\{u_{ij}\geq u_{ik},\forall k\}
    \end{aligned}
    \nonumber
\end{equation}
The consumer-specific choice probabilities are then given by
\begin{equation}
    \begin{aligned}
        s_{ij}:=\mathbb{E}[q_{ij}\mid J_i,\chi_i]=\int_{\mathcal{A}_{ij}}d F_u\left(u_{i0},...,u_{iJ_i}\mid J_i, \chi_i\right)
    \end{aligned}
    \nonumber
\end{equation}
where $\mathcal{A}_{ij}=\{(u_{i0},...,u_{iJ_i})\in\mathbb{R}^{J_i+1}\mid u_{ij}\geq u_{ik},\forall k\}$.


\section{The Canonical Model}
\begin{definition}[Canonical Model]
    Discrete choice demand models are frequently formulated using a parametric random utility specification such as
    \begin{equation}
        \begin{aligned}
            u_{ijt}=x_{jt}\beta_{it}-\alpha_{it}p_{jt}+\xi_{jt}+\epsilon_{ijt}
        \end{aligned}
        \nonumber
    \end{equation}
    for $j>0$, with $u_{i0t}=\epsilon_{i0t}$.
\end{definition}

The notion of a “market” $t$ allows a precise characterization of the endogeneity problems inherent to demand estimation (In practice, markets are typically defined by natural combinations of time and geography). Let $\mathcal{J}_t$ denote the set of products (inside goods) available to consumers in market $t$, and let $J_t=|\mathcal{J}_t|$.

Let $x_t=(x_{1t},...,x_{J_t,t}), p_t=(p_{1t},...,p_{J_t,t}), \xi_t=(\xi_{1t},..., \xi_{J_t,t})$, and $\chi_t=(x_t,p_t,\xi_t)$.

\begin{enumerate}
    \item $p_{jt}$ represents the price of good $j$ in market $t$, while $x_{jt}\in \mathbb{R}^K$ represents other observable characteristics of good $j$ in the market.
    \item $\xi_{jt}$ is \textit{ a demand shock}, an unobserved factor associated with good $j$ and market $t$.
    \subitem $\circ$ ($\xi_{jt}$ can represent any combination of latent taste variation and latent product characteristics common to consumers in market $t$. For example, a high value of $\xi_{jt}$ may simply indicate that consumers in market $t$ have a high mean taste for good $j$.)
    \subitem $\circ$ ($\xi_{jt}$ is correlated with $p_{jt}$ and $x_{jt}$ by the endogeneity of prices and additional characteristics.)
    \item $\epsilon_{ijt}$ is the \textit{utility shock}. It is most often specified as an i.i.d. draw from a standard type-1 extreme value distribution, yielding a \textit{mixed multinomial logit model}.
    \subitem $\circ$ Choice probabilities in the population reflect a mixture of the choice probabilities conditional on each possible combination of ($\alpha_{it},\beta_{it}$). In this case, the choice probabilities in the population (i.e., the market shares) are given by
    \begin{equation}
        \begin{aligned}
            s_{jt}=\int \frac{e^{x_{jt}\beta_{it}-\alpha_{it}p_{jt}+\xi_{jt}}}{\sum_{k=0}^{J_t}e^{x_{kt}\beta_{it}-\alpha_{it}p_{kt}+\xi_{kt}}}d F(\alpha_{it},\beta_{it};t)
        \end{aligned}
        \label{eq:mixed_multinomial_logit}
    \end{equation}
    where the latent taste parameters $\alpha_{it}$ and $\beta_{it}$ are often referred to as ``random coefficients,'' and $F(\cdot;t)$ denotes their joint distribution in market $t$.
    \subitem $\circ$ Alternatively, a normal distribution will yield a mixed multinomial probit.
    \item The joint distribution $F(\cdot;t)$ is commonly specified as follows.
    \begin{enumerate}
        \item Each component $k$ of the random coefficient vector $\beta_{it}$ typically specified takes the form
        \begin{equation}
            \begin{aligned}
                \beta_{it}^{(k)}=\beta_{0}^{(k)}+\beta_{v}^{(k)}v_{it}^{(k)}+\sum_{l=1}^L\beta_{d}^{(l,k)}d_{ilt}
            \end{aligned}
            \label{eq:beta_it}
        \end{equation}
        Remind that the $\beta_{it}^{(k)}$ contributes as $x_{jt}^{(k)}\beta_{it}^{(k)}$.
        \begin{enumerate}
            \item The $\beta_{0}^{(k)}$ is a parameter shifting \underline{all} consumers' tastes for additional characteristics $x_{jt}^{(k)}$.
            \item Each $d_{ilt}$ represents a \textit{characteristic} (e.g., demographic measure) of individual $i$; the $\beta_{d}^{(l,k)}$ governs the extent of variation in tastes for $x_{jt}^{(k)}$ with different values of $d_{ilt}$.
            \item Each $v_{it}^{(k)}$ represents a \textit{taste shock}, which is a random variable with a pre-specified distribution (e.g., a standard normal); the $\beta_{v}^{(k)}$ governs the extent of variation in tastes for $x_{jt}^{(k)}$ with different values of $v_{it}^{(k)}$.
        \end{enumerate}
        \item A typical specification of $\alpha_{it}$ takes the form
        \begin{equation}
            \begin{aligned}
                \ln\left(\alpha_{it}\right)=\alpha_0+\alpha_y y_{it}+\alpha_v v_{it}^{(0)}
            \end{aligned}
            \label{eq:alpha_it}
        \end{equation}
        where $y_{it}$ represents consumer-specific measures such as income that are posited to affect price sensitivity. The variables included in $y_{it}$ might overlap partially or entirely with $d_{it}$.
    \end{enumerate}
\end{enumerate}


\section{Market-Level Data}
We typically observe key data at the market level:
\begin{enumerate}
    \item $J_t$: the number of goods available to consumers in each market $t$;
    \item $p_t,x_t$: prices and additional characteristics of goods in each market $t$;
    \item $\tilde{s}_{jt}$: observed market shares, typically measured by $\tilde{s}_{jt}:=\frac{\textnormal{total quantity of good $j$ sold in market $t$}}{\textnormal{total number of consumers in market $t$}}$.
    \item \textit{Distributions of $(d_{it},y_{it})$}: distribution of consumer characteristics in each market.
    \item Additional variables $w_t$ that might serve as appropriate instruments.
\end{enumerate}


\section{The BLP Estimator}
We follow \cite{berry1995automobile} in assuming that each $\epsilon_{ijt}$ is an i.i.d. draw from a standard type-1 extreme value (Gumbel) distribution and that each $v_{it}^{(k)}$ in \eqref{eq:beta_it} is an i.i.d. draw from a standard normal distribution.

Let $$\theta\equiv(\alpha_0,\beta_0,\beta_d,\beta_v)$$
represent all the parameters of the model. It will be useful to partition these as
\begin{equation}
    \begin{aligned}
        \theta_1=(\alpha_0,\beta_0),\quad \theta_2=(\beta_d,\beta_v)
    \end{aligned}
    \nonumber
\end{equation}





\bibliography{ref}




\end{document}