\documentclass[11pt]{elegantbook}
\usepackage{graphicx}
%\usepackage{float}
\definecolor{structurecolor}{RGB}{40,58,129}
\linespread{1.6}
\setlength{\footskip}{20pt}
\setlength{\parindent}{0pt}
\newcommand{\argmax}{\operatornamewithlimits{argmax}}
\newcommand{\argmin}{\operatornamewithlimits{argmin}}
\elegantnewtheorem{proof}{Proof}{}{Proof}
\elegantnewtheorem{claim}{Claim}{prostyle}{Claim}
\DeclareMathOperator{\col}{col}
\title{Time Series}
\author{Wenxiao Yang}
\institute{Haas School of Business, University of California Berkeley}
\date{2024}
\setcounter{tocdepth}{2}
\extrainfo{All models are wrong, but some are useful.}

\cover{cover.png}

% modify the color in the middle of titlepage
\definecolor{customcolor}{RGB}{32,178,170}
\colorlet{coverlinecolor}{customcolor}
\usepackage{cprotect}

\addbibresource[location=local]{reference.bib} % bib

\begin{document}
\maketitle

\frontmatter
\tableofcontents

\mainmatter

\chapter{Time Series Analysis}
\section{Goals and Terminology}
\textbf{Data} in time series is denoted by
\begin{equation}
    \begin{aligned}
        \{\underbrace{y_t}_{n\times 1}:1\leq t\leq T\}
    \end{aligned}
    \nonumber
\end{equation}
%Some fundamental assumptions are needed for statistics.
\begin{assumption}
    Each $y_t$ is the realization of some random vector $Y_t$.
\end{assumption}
The \textbf{objective} is to provide data-based answers to questions about the distribution of $\{Y_t:1\leq t\leq T\}$.

The \textbf{challenge} we face is $Y_1,Y_2,...,Y_T$ are \textit{not necessarily independent}. Time series analysis gives the models and methods that can accommodate dependence.

Some terminologies we need to know:
\begin{definition}[Stochastic Process]
    A \textbf{stochastic process} is a collection $\{Y_t:1\leq t\leq T\}$ of random variables/vectors (defined on the same probability space).
    \begin{enumerate}
        \item $\{Y_t:1\leq t\leq T\}$ is \textbf{discrete time process} if $\mathcal{T}=\{1,...,T\}$ or $\mathcal{T}=\mathbb{N}=\{1,2,...\}$ or $\mathcal{T}=\mathbb{Z}=\{...,-1,0,1,...\}$
    \end{enumerate}
\end{definition}


\end{document}