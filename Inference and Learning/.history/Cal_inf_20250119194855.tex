\documentclass[11pt]{elegantbook}
\usepackage{graphicx}
%\usepackage{float}
\definecolor{structurecolor}{RGB}{40,58,129}
\linespread{1.6}
\setlength{\footskip}{20pt}
\setlength{\parindent}{0pt}
\newcommand{\argmax}{\operatornamewithlimits{argmax}}
\newcommand{\argmin}{\operatornamewithlimits{argmin}}
\elegantnewtheorem{proof}{Proof}{}{Proof}
\elegantnewtheorem{claim}{Claim}{prostyle}{Claim}
\DeclareMathOperator{\col}{col}
\title{Causal Inference}
\author{Wenxiao Yang}
\institute{Haas School of Business, University of California Berkeley}
\date{2024}
\setcounter{tocdepth}{2}
\extrainfo{All models are wrong, but some are useful.}

\cover{cover.png}

% modify the color in the middle of titlepage
\definecolor{customcolor}{RGB}{32,178,170}
\colorlet{coverlinecolor}{customcolor}
\usepackage{cprotect}


%\bibliographystyle{apalike_three}

\begin{document}
\maketitle

\frontmatter
\tableofcontents

\mainmatter



\chapter{Causal Inference}
The fundamental problem of causal inference:
\begin{enumerate}[(a).]
    \item Never see the same person treated and untreated
    \item Missing data problem
    \item "Solve" by finding a comparison group
\end{enumerate}

\begin{definition}[Notations and Estimands]
    \normalfont
    \begin{enumerate}[$\circ$]
        \item Treatment: $T\in\{0,1\}$
        \item Potential Outcome with treatment $Y(1), Y(0)$
        \item Other Variable $X$
        \item Individual Treatment Effect (ITE) $= Y_i(1)-Y_i(0)$
        \item Conditional Average Treatment Effect (CATE) $=\mathbb{E}[Y(1)-Y(0)|X=x]:=\tau(x)$
        \item Average Treatment Effect (ATE) $=\mathbb{E}[Y(1)-Y(0)]:=\tau$
        \item Average Treatment Effects on Treated (ATT) $=\mathbb{E}[Y(1)-Y(0)\mid T=1]$
    \end{enumerate}
\end{definition}


\subsection*{Difference in Means}
\begin{equation}
    \begin{aligned}
        \hat{\tau}=\bar{Y}_1-\bar{Y}_0=\frac{1}{n_1}\sum_{i=1}^n Y_i T_i - \frac{1}{n_0}\sum_{i=1}^n Y_i(1-T_i)
    \end{aligned}
    \nonumber
\end{equation}

By the Law of Large Numbers,
\begin{equation}
    \begin{aligned}
        \lim_{n \rightarrow \infty}\frac{1}{n_1}\sum_{i=1}^n Y_i T_i&=\lim_{n \rightarrow \infty}\frac{n}{n_i}\frac{1}{n}\sum_{i=1}^n Y_i T_i\\
        &=\left(P[T=1]\right)^{-1} \mathbb{E}[YT]\\
        &= \left(P[T=1]\right)^{-1}\mathbb{E}[YT\mid T=1]P[T=1]\\
        &=\mathbb{E}[YT\mid T=1]\\
        \bar{Y}_1 &\stackrel{P}{\longrightarrow} \mathbb{E}[YT\mid T=1]
    \end{aligned}
    \nonumber
\end{equation}

\subsubsection*{Causal Effect}
\begin{assumption}\quad
    \begin{enumerate}[(1).]
        \item SUTVA: Only your treatment matters;
        \item Consistency: Observed outcome matches treatment "assignment": $Y=TY(1)+(1-T)Y(0)$.
    \end{enumerate}
\end{assumption}

Only yields $\hat{\tau}=\bar{Y}_1-\bar{Y}_0\stackrel{P}{\longrightarrow} \mathbb{E}[Y(1)\mid T=1]-\mathbb{E}[Y(0)\mid T=0]$

\begin{equation}
    \begin{aligned}
        &\mathbb{E}[Y(1)\mid T=1]-\mathbb{E}[Y(0)\mid T=0]\\
        =&\underbrace{\mathbb{E}[Y(1)\mid T=1]-\mathbb{E}[Y(0)\mid T=1]}_{\textnormal{ATT}}+\underbrace{\mathbb{E}[Y(0)\mid T=1]-\mathbb{E}[Y(0)\mid T=0]}_{\textnormal{selection bias}}\\
    \end{aligned}
    \nonumber
\end{equation}
To get the ATT (eliminate the selection bias), we need exclusion/independence: Randomization.


Assume $Y(t)=\mu(t)+\epsilon_t$ (SUTVA). Consider the consistency assumption:
\begin{equation}
    \begin{aligned}
        Y&=TY(1)+(1-T)Y(0)\\
        &=Y(0)+T(Y(1)-Y(0))\\
        &=\underbrace{\mu_0}_\alpha+T\underbrace{(\mu_1-\mu_0)}_{\beta^T}+\underbrace{\epsilon_0+T(\epsilon_1-\epsilon_0)}_{\epsilon}
    \end{aligned}
    \nonumber
\end{equation}

Consider the covariate between $T$ and $X$. Why important?

\begin{equation}
    \begin{aligned}
        &\mathbb{E}[Y|X=1,T=1]-\mathbb{E}[Y|X=1,T=0]\\
        =& \underbrace{\mathbb{E}[Y(1)|X(1)=1]-\mathbb{E}[Y(0)|X(1)=1]}_{\textnormal{ATE}|X(1)=1} + \underbrace{\mathbb{E}[Y(0)|X(1)=1]-\mathbb{E}[Y(0)|X(0)=1]}_{\textnormal{selection bias}}\\
    \end{aligned}
    \nonumber
\end{equation}


$Y(t)=\mu(t,X)+\epsilon_t$. Then,
\begin{equation}
    \begin{aligned}
        Y&=Y(1)T+Y(0)(1-T)\\
        &=\underbrace{\mu(0,X)}_{\alpha(X)} + T\underbrace{(\mu_1(X)-\mu_0(X))}_{\beta(X)}+\epsilon
    \end{aligned}
    \nonumber
\end{equation}









\bibliography{ref}

\end{document}