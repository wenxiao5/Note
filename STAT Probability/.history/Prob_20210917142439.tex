%!TEX program = xelatex
\documentclass[11pt,a4paper]{article}
\usepackage[utf8]{inputenc}
\usepackage[T1]{fontenc}
\usepackage{authblk}
\usepackage{ctex}
\usepackage{tikz}
\usepackage{pgfplots}
\usepackage{verbatim}
\usepackage{amsfonts}
\usepackage{amsmath}
\usepackage{amsthm}
\usepackage{indentfirst}
\usepackage{amssymb}
\setlength{\parindent}{0pt}
\usetikzlibrary{shapes,snakes}
\newcommand{\argmax}{\operatornamewithlimits{argmax}}
\newcommand{\argmin}{\operatornamewithlimits{argmin}}
\DeclareMathOperator{\col}{col}
\usepackage{booktabs}
\newtheorem{theorem}{Theorem}
\newtheorem{note}{Note}
\newtheorem{definition}{Definition}
\newtheorem{proposition}{Proposition}
\newtheorem{lemma}{Lemma}
\newtheorem{example}{Example}
\newtheorem{corollary}{Corollary}
\usepackage{graphicx}
\usepackage{geometry}
\usepackage{hyperref}
\newcommand{\code}{	exttt}
\geometry{a4paper,scale=0.8}
\title{Probability}
\author[*]{Wenxiao Yang}
\affil[*]{Department of Mathematics, University of Illinois at Urbana-Champaign}
\date{2021}


\begin{document}
\maketitle
\tableofcontents
\newpage


\section{Poisson Distribution $Pois(\lambda)$: 单位时间发生$k$次时间的概率}
\subsection{$\lambda$:单位时间发生该时间的平均次数}
$$\Pr(X{=}k)= \frac{\lambda^k e^{-\lambda}}{k!},\ k=0,1,2,3...$$
\subsection{推导}
我们考虑一段时间(讲单位时间微分成n等分, $n \rightarrow \infty$),每一刻(瞬间)都有一个event may occur, which follows binomial distribution $B(n,p)$. where $n \rightarrow \infty, p \rightarrow 0$; $\lambda=n\cdot p$ is the expected number of events in this period of time.\\
































\section{Exponential distribution $Exp(\lambda)$: 独立随机事件的发生间隔}
\subsection{$\lambda$:单位时间发生该时间的平均次数}
随机变量$X$服从参数为$\lambda$ 或$\beta$ 的指数分布,则记作
$${\displaystyle X\sim {\text{Exp}}(\lambda )} \text{ or } {\displaystyle X\sim {\text{Exp}}(\beta )}$$
两者意义相同,只是$\lambda$ 与$\beta$ 互为倒数关系.
$${f(x;{\lambda})=\left\{{\begin{matrix}{\lambda }e^{-{\lambda }x}&x\geq 0\\0&\;x<0.\end{matrix}}\right.}$$
$${f(x;{\beta})=\left\{{\begin{matrix}{\frac{1}{\beta} }e^{-{\frac{1}{\beta} }x}&x\geq 0\\0&\;x<0.\end{matrix}}\right.}$$
累积分布函数为:
$${F(x;{\lambda})=\left\{{\begin{matrix}{1-}e^{-{\lambda }x}&x\geq 0\\0&\;x<0.\end{matrix}}\right.}$$
其中$\lambda > 0$是分布的参数,即每单位时间发生该事件的次数; $\beta>0$ 为尺度参数,即该事件在每单位时间内的发生率。两者常被称为率参数(rate parameter)。指数分布的区间是$[0,\infty)$。

\subsection{$\mathbb{E}(X)=\frac{1}{\lambda}$:预期事件的发生间隔; $Var(X)=\frac{1}{\lambda^2}$}
$$\mathbb{E}(X)=\frac{1}{\lambda};\ Var(X)=\frac{1}{\lambda^2}$$
\subsection{Memorylessness: ${\displaystyle \Pr \left(T>s+t\mid T>s\right)=\Pr(T>t)}$}
\begin{equation}
    \begin{aligned}
        \Pr (T>s+t\mid T>s)&=\frac{\Pr(T>s+t\text{ and }T>s)}{\Pr(T>s)}\\
        &=\frac{\Pr(T>s+t)}{\Pr(T>s)}\\
        &=\frac{e^{-\lambda(s+t)}}{e^{-\lambda s}}\\
        &=e^{-\lambda t}\\
        &=\Pr (T>t)
    \end{aligned}
    \nonumber
\end{equation}



















































\end{document}