%!TEX program = xelatex
\documentclass[11pt,a4paper]{article}
\usepackage[utf8]{inputenc}
\usepackage[T1]{fontenc}
\usepackage{authblk}
\usepackage{ctex}
\usepackage{tikz}
\usepackage{pgfplots}
\usepackage{verbatim}
\usepackage{amsfonts}
\usepackage{amsmath}
\usepackage{amsthm}
\usepackage{indentfirst}
\usepackage{amssymb}
\setlength{\parindent}{0pt}
\usetikzlibrary{shapes,snakes}
\newcommand{\argmax}{\operatornamewithlimits{argmax}}
\newcommand{\argmin}{\operatornamewithlimits{argmin}}
\DeclareMathOperator{\col}{col}
\usepackage{booktabs}
\newtheorem{theorem}{Theorem}
\newtheorem{note}{Note}
\newtheorem{definition}{Definition}
\newtheorem{proposition}{Proposition}
\newtheorem{lemma}{Lemma}
\newtheorem{example}{Example}
\newtheorem{corollary}{Corollary}
\usepackage{graphicx}
\usepackage{geometry}
\usepackage{hyperref}
\newcommand{\code}{	exttt}
\geometry{a4paper,scale=0.8}
\title{Analysis}
\author[*]{Wenxiao Yang}
\affil[*]{Department of Mathematics, University of Illinois at Urbana-Champaign}
\date{2022}
\begin{document}
\maketitle
\tableofcontents
\newpage

\section{Basis}
\subsection{Sequence Definitions}
Sequences $\left\{x_{k}\right\}_{k=1}, \ldots$ or $\left\{x_{k}\right\}, x_{k} \in \mathbb{R}^{n}$
\begin{definition}[Convergence: note $x_{k} \rightarrow x, \lim _{k \rightarrow \infty} x_{k}=x$]
    Given $\varepsilon>0, \quad \exists N_{\varepsilon}$ s.t. $$\quad\left\|x_{k}-x\right\|<\varepsilon \quad \forall k \geqslant N_{\varepsilon}$$
\end{definition}

\begin{definition}[Cauchy Sequence]
    $\{x_k\}$ is Cauchy if given $\varepsilon>0, \quad \exists N_{\varepsilon}$ s.t.
    $$\left\|x_{k}-x_{m}\right\|<\varepsilon,\  \forall k, m \geqslant N_{\varepsilon} \text {. }$$
\end{definition}
\textbf{Note:}$$\left\{x_{k}\right\} \text { converges } \Longleftrightarrow\left\{x_{k}\right\} \text { is Cauchy}$$


\begin{definition}[Subsequence]
Infinite subset of $\{x_k\}$: $\{x_k:k\in \mathcal{K} \}\text{ or } \{x_k\}_\mathcal{K} $, where $\mathcal{K} $ is subset of $\mathbb{Z}^+$.
\end{definition}

\begin{definition}[Limit point]
$x$ is a limit point of $\left\{x_{k}\right\}$ if $\exists \text { a subsequence of }\left\{x_{k}\right\} \text { that converges to } x$.
\end{definition}

\begin{definition}[Bounded Sequence]
    $$\left\|x_{k}\right\| \leqslant b, \forall k$$
\end{definition}

Results about Bounded sequences:

1. Every bounded has at least one limit point.

2. A bounded sequence converges iff it has a \textbf{unique limit point}.

\subsection{Scalar Sequences}
\textbf{\underline{Scalar sequences}} $\left\{x_{k}\right\}, x_{k} \in \mathbb{R}$:
\begin{proposition}
    If $\left\{x_{k}\right\}$ is bounded above(below) and non-decreasing(non-increasing) it \textbf{converges}.
\end{proposition}

\begin{proposition}
    The largest(smallest) limit point of $\left\{x_{k}\right\}$ is $\lim _{k \rightarrow \infty}\sup x_{k}$ ($\lim _{k \rightarrow \infty}\inf x_{k}$)
\end{proposition}

\begin{proposition}
    $\left\{x_{k}\right\}$ converges $\Longleftrightarrow-\infty<\lim _{k \rightarrow \infty} \inf x_{k}=\lim _{k \rightarrow \infty}\sup x_{k}<\infty$
\end{proposition}

\subsection{Functions Basis}
\begin{definition}[Continunity]
    A real-valued function $f$ is continuous at $x$ if
    
    for every $\left\{x_{k}\right\}$ converging to $x$ satisfies that $\lim _{k \rightarrow \infty} f\left(x_{k}\right)=f(x)$.

    Equivalent: given $\varepsilon>0, \exists \delta>0$ s.t.
    $|f(x)-f(y)|<\varepsilon \quad \forall\|y-x\|<\delta$

    $f$ is continuous if it is continuous at all points $x$.
\end{definition}

\begin{definition}[Coercive]
    A real-valued function $f:\& \rightarrow \mathbb{R}$ is \underline{coercive} if for \textbf{every} $\left\{x_{k}\right\} \subset \&$ s.t. $\left\|x_{k}\right\| \rightarrow \infty, f\left(x_{k}\right) \rightarrow \infty$
\end{definition}

\begin{example}[Check coercive]
\end{example}
1) $x \in \mathbb{R}^{2}, f(x)=x_{1}^{2}+x_{2}^{2}$ - coercive

2) $x \in \mathbb{R}, f(x)=1-e^{-|x|}$ - not coercive

3) $x \in \mathbb{R}^{2}, f(x)=x_{1}^{2}+x_{2}^{2}-2 x_{1} x_{2}$ - not coercive
(需要所以$\left\|x_{k}\right\| \rightarrow \infty$都满足)


\subsection{Sets}
\begin{definition}[Open Sets]
    A set $\& \subseteq \mathbb{R}^{n}$ is open if
    
    $\forall x \in \&$ we can draw a ball around $x$ that is contained in $\&$.

    i.e. $\forall x \in \&, \exists \varepsilon>0$ s.t. $\{y:\|y-x\|<\varepsilon\} \leq-\sigma$
\end{definition}

\begin{definition}[Closed Sets]
    $\&$ is closed if $\&^c$ is open

    Equivalent: if $\&$ contains all limit points of all sequences in $\&$
\end{definition}
\begin{example}[Closed and Open Sets]
\end{example}
1) $(1,2)=\{x \in \mathbb{R}: 1<x<2\}$ - open

2) $\mathbb{R}$ is both open and closed

3) $(-\infty, 1)=\{x \in \mathbb{R}: x<1\}$ - open

4) $[1, \infty)$ is closed because its complement open

5) $(1,2]$ is neither open nor closed

\begin{definition}[Bounded Set]
    $A$ is bounded if $\exists M$ s.t. $\|x\| \leqslant M \quad \forall x \in\&$
\end{definition}

\begin{definition}[Compact Set]
    $\mathcal{L} \subseteq \mathbb{R}^{n}$ is compact of it is closed and bounded.
\end{definition}

\begin{example}[Compact Set]
    $[1,2]=\{x \in \mathbb{R}: 1 \leqslant x \leqslant 2\}$; $\left\{x \in \mathbb{R}^{2}\right.: \left.x_{1}^{2}+x_{2}^{2} \leqslant 4\right\}$
\end{example}

\begin{definition}[Extreme of sets of scalars, $\sup A,\inf A$]
    Let $A\subset \mathbb{R}$.

    - The infimum of A, or inf A is largest $y$ s.t. $y \leqslant x, \forall x \in A$. If no such $y$ exists, $\inf A=-\infty$

    - Similar definition for supremum of $A$ (or wrote as $\sup A$).
\end{definition}
\begin{proposition}
    If $\inf A(\sup A)=x^*\in A$, then $x^*=\min A(\max A)$
\end{proposition}

\section{Functions}
\subsection{Extreme of Functions}
\begin{definition}[Extreme of Functions]
    Let $\& \subseteq \mathbb{R}^{n}, f: \& \rightarrow \mathbb{R}$
    $$\inf_{x \in \&} f(x)=\inf\{f(x): x \in \&\}$$
\end{definition}

If $\exists x^{*} \in \& \text { s.t. inf } f(x)=f\left(x^{*}\right)$. Then, $f$ achieves (attains) its minimum and $f\left(x^{*}\right)=\min _{x \in \&} f(x)$

$x^{*}$ is called a \textbf{minimizer} of $f$, written as $x^{*} \in \arg \min _{x \in \&} f(x)$. If $x^*$ is uniqne, we write $x^{*}=\arg \min _{x \in \&} f(x)
$

Similarly, supremum and maximum of $f$.

\subsubsection{Weierstrass' Theorem(Extreme value Theorem)}
\begin{theorem}
    [Weierstrass' Theorem(Extreme value Theorem)]
    
    If $f$ is a \textbf{continuous} function on a \textbf{compact set}, $\& \subseteq \mathbb{R}^{n}$, then $f$ attains its min and max on $\&$ i.e.,
    $$
    \begin{aligned}
    \exists x_1 \in \& \text { s.t. } f\left(x_{1}\right) &=\inf _{x \in \&} f(x) \\
    \exists x_{2} \in \& \text { s.t. } f\left(x_{2}\right) &=\sup _{x \in \&} f(x)
    \end{aligned}
    $$
\end{theorem}
\begin{proof}
    (for existence of $\min$; $\max$ is similar)

    Let $\{\sigma_k\}\subseteq \&$ be s.t.
    $$\inf_{x\in\&} f(x) \leq f\left(\sigma_{k}\right) \leq \inf _{x \in \&} f(x)+\frac{1}{k}$$

    Then $\lim _{k \rightarrow \infty} f\left(\sigma_{k}\right)=\inf_{x\in\&} f(x)$

    $\mathcal{L}$ is bounded $\Rightarrow\left\{\sigma_{k}\right\}$ has it least one limit point $x$,

    $\mathcal{L}$ is closed $\Rightarrow x_{1} \in \&$

    $f$ is continuous $\Rightarrow f\left(x_{1}\right)=\lim _{k \rightarrow \infty} f\left(\sigma_{k}\right)=\inf _{x \in \&} f(x)$
\end{proof}


Corollary Let $f$ be continuous on closed set 1 , that is not necessarily bounded. If $f$ is coercive on 1 it attains its $\min _{(\max )}$ on s.

Proof Consider $\left\{z_{k}\right\}$ as in proof of $W T$.

$f$ is coercive on $\mathcal{L} \Rightarrow\left\{z_{k}\right\}$ is bounded

Rest of proof same as proof $\mathrm{O} \mathrm{WT}$. Example

$f(x)=f\left(x_{1}, x_{2}, x_{3}\right)=x_{1}^{4}+2 x_{2}^{2}+e^{-x_{3}}+e^{2 x_{3}}$

(a) Dos $f$ achieve its min. and max. on

$$
\mathcal{L}_{1}=\left\{x \in \mathbb{R}^{3}: x_{1}^{2}+2 x_{2}^{2}+3 x_{3}^{2} \leqslant 6\right\} \text { ? }
$$

$\mathcal{L}_{1}$ is compact and $f$ is continuows

$\Rightarrow$ both min and max achieved on s, by WT

(b) Does $f$ achieve its min and max over $\mathbb{R}^{3}$ ?

$f \rightarrow \infty$ whenever $\|x\| \rightarrow \infty \Rightarrow f$ is coercive

- $\mathbb{R}^{3}$ is closed

$\Rightarrow f$ achieves its min. on $\mathbb{R}^{3}$ by corollary to WT.

$\max$, dos not exist sunce $f \rightarrow \infty$ as $\|x\| \rightarrow \infty$.

(c) Does $f$ achieve its min and max orer:

%$$
%\mathscr{L}_{2}=\left\{x \in \mathbb{R}^%{3}: x_{1}+x_{2}+x_{3}=3\right\} \text {. %}
%$$

- $\mathcal{L}_{2}$ is closed, but not bounded.

For example, $x_{1}=0, x_{2}=3-x_{3}$ and $x_{3} \rightarrow \infty$

- Since $f$ is coercive, min achiered

- $\max$ does nat exist suice setting $x_{1}=0$

$x_{2}=3-x_{3}$ and letting $x_{3} \rightarrow \infty$ makes $f \rightarrow \infty$ Matrices

%$$
%A_{m \times n}=\left[\begin{array}{cccc}
%a_{11} & a_{12} & \cdots & a_{1 n} \\
%a_{21} & a_{22} & & a_{2 n} \\
%\vdots & a_{m 2} & a_{m n}
%\end{array}\right] \quad \begin{array}%{ll} 
%& a_{i j} \\
%a_{m 1} & a_{m 2} & \equiv & A_{i j} \\
%& \equiv & {[A]_{i j}}
%\end{array}
%$$

$$
\begin{aligned}
&\text { - Square matrix: } m=n, A_{n \times n} \\
&\text { - Symmetric: } A^{\top}=A \\
&\text { - Upper Triongular: } a_{i j}=0 \text { whenever } i<j \text {. }
\end{aligned}
$$


\end{document}