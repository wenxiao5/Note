%!TEX program = xelatex
\documentclass[11pt,a4paper]{article}
\usepackage[utf8]{inputenc}
\usepackage[T1]{fontenc}
\usepackage{authblk}
\usepackage{ctex}
\usepackage{tikz}
\usepackage{pgfplots}
\usepackage{verbatim}
\usepackage{amsfonts}
\usepackage{amsmath}
\usepackage{amsthm}
\usepackage{indentfirst}
\usepackage{amssymb}
\setlength{\parindent}{0pt}
\usetikzlibrary{shapes,snakes}
\newcommand{\argmax}{\operatornamewithlimits{argmax}}
\newcommand{\argmin}{\operatornamewithlimits{argmin}}
\DeclareMathOperator{\col}{col}
\usepackage{booktabs}
\newtheorem{theorem}{Theorem}
\newtheorem{note}{Note}
\newtheorem{definition}{Definition}
\newtheorem{proposition}{Proposition}
\newtheorem{lemma}{Lemma}
\newtheorem{example}{Example}
\newtheorem{corollary}{Corollary}
\usepackage{graphicx}
\usepackage{geometry}
\usepackage{hyperref}
\newcommand{\code}{	exttt}
\geometry{a4paper,scale=0.8}
\title{Analysis}
\author[*]{Wenxiao Yang}
\affil[*]{Department of Mathematics, University of Illinois at Urbana-Champaign}
\date{2022}
\begin{document}
\maketitle
\tableofcontents
\newpage

\section{Sequence}
Sequences $\left\{x_{k}\right\}_{k=1}, \ldots$ or $\left\{x_{k}\right\}, x_{k} \in \mathbb{R}^{n}$

$$
\text { Definition (convergence) } \quad x_{k} \rightarrow x, \lim _{k \rightarrow \infty} x_{k}=x
$$

Given $\varepsilon>0, \quad \exists N_{\varepsilon}$ s.t. $\quad\left\|x_{k}-x\right\|<\varepsilon \quad \forall k \geqslant N_{\varepsilon}$

$$
\text { Definition (Cauchy segmence) }\left\{x_{k}\right\} \text { is Canchy if }
$$\[
\begin{array}{l}
$$\text { given } \varepsilon>0, \exists N_{\varepsilon} s \cdot t .\left\|x_{k}-x_{m}\right\|<\varepsilon \forall k, m \geqslant N_{\varepsilon} \text {. }$$ \\
$$\left\{x_{k}\right\} \text { converges } \Longleftrightarrow\left\{x_{k}\right\} \text { is Canchy }$$
\end{array}
\]

$$
\begin{aligned}
&\text { Definition (subsequence) infinite subset of }\left\{x_{k}\right\} \text {. } \\
&\left\{x_{k}: k \in X\right\} \text { or }\left\{x_{k}\right\} K \text {, } \text { K:\mathrm{ infinite subst of } \mathbb { Z } ^ { + } }
\end{aligned}
$$

$$
\text { Definition (Limit point) } x \text { is a limit point of }\left\{x_{k}\right\}
$$\[
\begin{array}{l}
$$\text { if } \exists \text { a subseguence of }\left\{x_{k}\right\} \text { thet converges to } x \text {. }$$
\end{array}
\]

$$
\text { Difinition (bounded pequence) }\left\|x_{k}\right\| \leqslant b, \forall k
$$

$$
\begin{aligned}
&\text { Results about Bounded sequences } \\
&\text { 1. Every bounded has at least one limit point } \\
&\text { 2. A bounded sequence converjes iff it has a } \\
&\text { unime limit point }
\end{aligned}
$$

Scalar sequences $\left\{x_{k}\right\}, x_{k} \in \mathbb{R}$

(below) (increasing)
above and non-decreasing

- If $\left\{x_{k}\right\}$ is bounded above and non-decreasing it Converges

- The largest limit point of $\left\{x_{k}\right\}$ is $\lim _{k \rightarrow \infty}$ sup $x_{k}$

- $\left\{x_{k}\right\}$ lonvergos $\Longleftrightarrow-\infty<\lim _{k \rightarrow \infty} \inf _{k}=\lim _{k \rightarrow \infty} x_{k}<\infty$

D)efinition (contimuity) A real-valued function $f$ is Continnous at $x$ if for every $\left\{x_{k}\right\}$ converging to $x$ $\lim _{k \rightarrow \infty} f\left(x_{k}\right)=f(x)$

Equivalenty, given $\varepsilon>0, \exists \delta>0$ s.t.

$|f(x)-f(y)|<\varepsilon \quad \forall\|y-x\|<\delta$

$f$ is contimuons if it is contimuons at all points $x$

Definition (coercive) A real-valued function $f: \mathcal{A} \rightarrow \mathbb{R}$ is coercive if for every $\left\{x_{k}\right\} \subset \&$ s.t. $\left\|x_{k}\right\| \rightarrow \infty, f\left(x_{k}\right) \rightarrow \infty$

Examples 1) $x \in \mathbb{R}^{2}, f(x)=x_{1}^{2}+x_{2}^{2}-$ coercive

2) $x \in \mathbb{R}, f(x)=1-e^{-|x|}-$ not coercive

3) $x \in \mathbb{R}^{2}, f(x)=x_{1}^{2}+x_{2}^{2}-2 x_{1} x_{2}$

$=\left(x_{1}-x_{2}\right)^{2}-$ not coercive Closed and open sets

A set \& $\subseteq \mathbb{R}^{n}$ is open if $\forall x \in \&$ we can dran a ball around $x$ that is contanicd in $\&$,

le. $\forall x \in \&, \exists \varepsilon>0$ s.t.

$\{y:\|y-x\|<\varepsilon\} \leq-8$

$+\frac{1}{1} z$

\& is closed if \&c is open

\equiv if \& contans all limit points of all sequences in \&

Examples $(1,2)=\{x \in \mathbb{R}: 1<x<2\}-$ open

$\mathbb{R}$ is both

$(-\infty, 1)=\{x \in \mathbb{R}: x<1\}-$ open

open and closed $[1, \infty)$ is closed because complement open

$(1,2]$ is neither open nor closed

Compact Set $\mathcal{L} \subseteq \mathbb{R}^{n}$ is compact of it is closed and bermded. A is bounded if $\exists M$ s.t. $\|x\| \leqslant M \quad \forall x \in-\&$

Examples

$[1,2]=\{x \in \mathbb{R}: 1 \leqslant x \leqslant 2\}$

$\left\{x \in \mathbb{R}^{2}\right.$
of scalars

$\left.x_{1}^{2}+x_{2}^{2} \leqslant 4\right\}$

Extrema of sets of scalars Let A $C \mathbb{R}$.

- The infinum of A, or inf A is largest $y$ s.t. $y \leqslant x, \forall x \in A$. If no such $y$ exists, inf $A=-\infty$

- Similar definition for supremum of $f$ or sup $A$.

- If $\operatorname{mif} A=x^{*}$ for some $x^{*} \in A$, then

$x^{*}=$ min $A$ or minimem of $A$.

- similarly max A or maximum of A. Examples 1) $\begin{aligned} A=[1,2], & \text { inf } A=\min A=1 \\ \text { anp } A=\max A=2 \end{aligned}$

$$
\begin{array}{ll}
\text { 2) } A=(1,2], & \text { inf } A=1, \text { not achieved } \\
\text { 3) } A=(1, \infty), & \text { sup } A=m x A=2 \\
\text { sup } A=\text { no maximum } &
\end{array}
$$

Extrema I Functiono Let $\ \subseteq \mathbb{R}^{n}, f: \& \rightarrow \mathbb{R}$

$$
\begin{aligned}
&\operatorname{inif}_{x \in \&} f(x)=\operatorname{mif}\{f(x): x \in \&\} . \\
&\text { If } \exists x^{*} \in \& \text { s.t. inf } f(x)=f\left(x^{*}\right) \text { Then }
\end{aligned}
$$

$f$ achieves (attains) its minimum and

$$
f\left(x^{*}\right)=\min _{x \in \mathcal{L}} f(x)
$$

$x^{*}$ is called a minimizer of $f$, written as

$$
x^{*} \in \arg \min _{x \in \mathcal{L}} f(x) } \\{\text { If } x^{*} \text { is unigne we write } x^{*}=\arg \min _{x \in s} f(x) \text {. }}
$$

Similarly, supremum and maximum of $f$.

$$
\text { Example 1 } \begin{aligned}
& f(x)=x, \quad x \in(-1,2) \\
& \text { sup } f(x)=2 \text {, neither achleved } \\
& \text { inf } f(x)=-1
\end{aligned}
$$

Exanple 2 $f(x)=x^{2}, x \in \mathbb{R}$

$$
\begin{aligned}
&f(x)=x^{2}, x \in \mathbb{R} \\
&\text { inf } f(x)=\min f(x)=0, x^{*}=0 \\
&\text { sip } f(x)=\infty, \text { mox doen wot exist. }
\end{aligned}
$$


\end{document}