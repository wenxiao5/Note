%!TEX program = xelatex
\documentclass[11pt,a4paper]{article}
\usepackage[utf8]{inputenc}
\usepackage[T1]{fontenc}
\usepackage{authblk}
\usepackage{ctex}
\usepackage{tikz}
\usepackage{pgfplots}
\usepackage{verbatim}
\usepackage{amsfonts}
\usepackage{amsmath}
\usepackage{amsthm}
\usepackage{indentfirst}
\usepackage{amssymb}
\setlength{\parindent}{0pt}
\usetikzlibrary{shapes,snakes}
\newcommand{\argmax}{\operatornamewithlimits{argmax}}
\newcommand{\argmin}{\operatornamewithlimits{argmin}}
\DeclareMathOperator{\col}{col}
\usepackage{booktabs}
\newtheorem{theorem}{Theorem}
\newtheorem{note}{Note}
\newtheorem{definition}{Definition}
\newtheorem{proposition}{Proposition}
\newtheorem{lemma}{Lemma}
\newtheorem{example}{Example}
\newtheorem{corollary}{Corollary}
\usepackage{graphicx}
\usepackage{geometry}
\usepackage{hyperref}
\newcommand{\code}{	exttt}
\geometry{a4paper,scale=0.8}
\title{STAT 425 Note 1}
\author[*]{Wenxiao Yang}
\affil[*]{Department of Mathematics, University of Illinois at Urbana-Champaign}
\date{2021}

\usepackage{listings}
\usepackage{xcolor}

\lstset{numbers=left,numberstyle=\tiny,keywordstyle=\color{blue},commentstyle=\color[cmyk]{1,0,1,0},frame=single,escapeinside=``,extendedchars=false,xleftmargin=2em,xrightmargin=2em,aboveskip=1em,tabsize=4,showspaces=false}






\begin{document}
\maketitle
\tableofcontents
\newpage

\section{Regression Analysis}
It is a "tool" used to examine the relationship between
a \textbf{Dependent Variable} or \textbf{Response} $Y$, and
one (or more) \textbf{Independent Variables} or \textbf{Regressors} or \textbf{Predictors} $X_1 ,X_2 ,...,X_p$.

\subsection{Simple Linear Regression}
$$y=\beta_0+\beta_1 x$$
$\beta_0$ is the *intercept*; $\beta_1$ is the *slope*.
One Response $\mathcal{Y}$; One Predictor $\mathcal{X}$
The data come in pairs:
$$\begin{aligned}
&x_1\quad &y_1\\&x_2\quad &y_2\\&\vdots\quad &\vdots\\&x_n\quad &y_n
\end{aligned}$$
$Y$ is a RANDOM VARIABLE that has a distribution for every level of the independent variable.

\subsection{Simple Linear Regression Model}
$$y_i=\beta_0+\beta_1 x_i+\varepsilon_i $$
where the *intercept* $\beta_0$, the *slope* $\beta_1$, and the *error variance* $\sigma^2$ are the *model parameters*.

\subsubsection{Assumptions of errors $\varepsilon$: 1. Mean zero, 2. umcorrelated, 3. homoscedastic}
The \textit{errors} $\varepsilon_1 , \varepsilon_2 , . . . , \varepsilon_n$ are assumed to\\
– have \textit{mean zero}: $E(\varepsilon_i ) = 0$\\
– be \textit{uncorrelated}: $Cov(\varepsilon_ i , \varepsilon_ j ) = 0, i \neq j$\\
– be \textit{homoscedastic}: $Var(\varepsilon_i ) = \sigma^ 2$ does not depend on $i$.

\subsubsection{Interpretation of $\beta_1$, $\beta_0$}
$\beta_1$ is the \textbf{change in the mean} of the probability distribution function of $y$ per unit change in $x$.\\
When $x=0$, $\beta_0$ is the \textbf{mean} of the probability distribution function of $y$(at $x=0$), otherwise $\beta_0$ \textbf{has no particular meaning}.\\

\subsection{Least Squares}
We want to find estimates of $\beta_0$, $\beta_1$ to minimize:
$$\min [y_i-E(y_i)]\Leftrightarrow \min [y_i-(\beta_0+\beta_1 x_i)]$$
minimize the \textbf{Residual Sum of Squares (RSS)}
$$RSS=\sum_{i=1}^n(y_i-\beta_0-\beta_1 x_i)^2$$
$$(\hat{\beta_0},\hat{\beta_1})=\argmin_{(\beta_0,\beta_1)}RSS$$

$$\begin{aligned}
    \frac{\partial RSS}{\partial \beta_0}=0 &\Leftrightarrow -2\sum_{i=1}^n(y_i-\beta_0-\beta_1 x_i)=0\\
    & \Leftrightarrow \beta_0 n+\beta_1\sum_{i=1}^n x_i=\sum_{i=1}^n y_i
\end{aligned}$$
$$\begin{aligned}
    \frac{\partial RSS}{\partial \beta_1}=0 &\Leftrightarrow -2\sum_{i=1}^n(y_i-\beta_0-\beta_1 x_i)x_i=0\\
    &\Leftrightarrow \beta_0 \sum_{i=1}^nx_i+\beta_1\sum_{i=1}^n x_i^2=\sum_{i=1}^n x_iy_i
\end{aligned}$$

\subsubsection{LS Estimators}
Then we can solve that
$$\begin{aligned}
&\hat{\beta}_{1}=\frac{\sum_{i=1}^{n}\left(x_{i}-\bar{x}\right)\left(y_{i}-\bar{y}\right)}{\sum_{i=1}^{n}\left(x_{i}-\bar{x}\right)^{2}}=\frac{\sum_{i=1}^{n} x_{i} y_{i}-n \bar{x} \bar{y}}{\sum_{i=1}^{n} x_{i}^{2}-n \bar{x}^{2}} \\
&\hat{\beta}_{0}=\bar{y}-\hat{\beta}_{1} \bar{x}
\end{aligned}$$

Alternative Representation of $\hat{\beta_1}$
$$\hat{\beta}_{1}=\frac{\sum_{i}\left(x_{i}-\bar{x}\right)\left(y_{i}-\bar{y}\right)}{\sum_{i}\left(x_{i}-\bar{x}\right)^{2}}=\frac{S_{x y}}{S_{x x}}=r_{x y} \sqrt{\frac{S_{y y}}{S_{x x}}}$$
Where 
\begin{equation}
    \begin{aligned}
        &S_{xy}=\sum_{i}\left(x_{i}-\bar{x}\right)\left(y_{i}-\bar{y}\right); &S_{xx}=\sum_{i}\left(x_{i}-\bar{x}\right)^{2}\\
        &S_{yy}=\sum_{i}\left(y_{i}-\bar{y}\right)^{2}; &r_{xy}=\frac{\sum_{i}\left(x_{i}-\bar{x}\right)\left(y_{i}-\bar{y}\right)}{\sqrt{\sum_{i}\left(x_{i}-\bar{x}\right)^{2}\sum_{i}\left(y_{i}-\bar{y}\right)^{2}}}=\frac{S_{xy}}{\sqrt{S_{xx}S_{yy}}}
    \end{aligned}
    \nonumber
\end{equation}

\subsubsection{Fitted Values \& Residuals}
The \underline{\textit{Prediction of $y_i$}} or the \underline{\textit{fitted value at $x_i$}}
$$\hat{y_i}=\hat{\beta_0}+\hat{\beta_1}x_i$$
$$\hat{y_i}=\bar{y}+\hat{\beta_1}(x_i-\bar{x})=\bar{y}+\frac{\sum_{i=1}^{n}\left(x_{i}-\bar{x}\right)\left(y_{i}-\bar{y}\right)}{\sum_{i=1}^{n}\left(x_{i}-\bar{x}\right)^{2}}(x_i-\bar{x})$$
The \underline{\textit{$i^{th}$ residual}}
$$r_i=y_i-\hat{y_i}$$

\subsubsection{Properties of residuals}
1. $\sum_i r_i=0$\\
2. $RSS=\sum_i r_i^2$ is minimized\\
3. $\sum_{i=1}y_i=\sum_{i=1}\hat{y_i}$\\
4. $\sum_ix_ir_i=0$ (proof: $\sum_ix_ir_i=\sum_ix_i(y_i-\bar{y}-\hat{\beta_1}(x_i-\bar{x}))=\sum_ix_iy_i-n\bar{x}\bar{y}-\frac{\sum_{i=1}^{n} x_{i} y_{i}-n \bar{x} \bar{y}}{\sum_{i=1}^{n} x_{i}^{2}-n \bar{x}^{2}}(\sum_ix_i^2-n\bar{x}^2)=0$)\\
5. $\sum_i \hat{y_i} r_i=0$ (inferred from 4)\\
6. The regression line always goes through the point $(bar{x},bar{y})$.

\subsubsection{Degree of freedom}
The \textbf{degree of freedom(df)} of the residuals is
$$df=\textit{(Sample size)}-\textit{(\# of parameters)}$$
$df=2$ in this case.

\subsubsection{Error variance}
The error variance is estimated by$$\hat{\sigma}^2=\frac{1}{n-2}\sum_i r_i^2$$

\subsection{Goodness of Fit: $R-$square}
\subsubsection{TSS, RSS, FSS}
$TSS: \sum_i(y_i-\bar{y})^2$\\
$RSS: \sum_ir_i^2$\\
$FSS: \sum_i(\hat{y}_i-\bar{y})^2$
$$\begin{aligned}
    \sum_{i}\left(y_{i}-\bar{y}\right)^{2} &=\sum_{i}\left(y_{i}-\hat{y}_{i}+\hat{y}_{i}-\bar{y}\right)^{2}=\sum_{i}\left(r_{i}+\hat{y}_{i}-\bar{y}\right)^{2} \\
    &=\sum_{i} r_{i}^{2}+\sum_{i}\left(\hat{y}_{i}-\bar{y}\right)^{2} \\
    T S S &=R S S+F S S
    \end{aligned}$$
\subsubsection{Cofficient of Determination($R^2$)}
$$R^{2}=\frac{\sum_{i}\left(\hat{y}_{i}-\bar{y}\right)^{2}}{\sum_{i}\left(y_{i}-\bar{y}\right)^{2}}=\frac{F S S}{T S S}=\frac{T S S-R S S}{T S S}=1-\frac{R S S}{T S S}$$
$0\leq R^2\leq 1$\\
It measures the effect of $X$ in reducing the variation in $Y$.\\
The larger $R^2$ is, the more the total variation of $y$ is reduced by reducing the independent variable $x$.\\

$R^2$ can also represent the degree of linear association between $X$ and $Y$.\\
\begin{equation}
    \begin{aligned}
       r_{xy}^2
       &=\frac{(\sum_{i}\left(x_{i}-\bar{x}\right)\left(y_{i}-\bar{y}\right))^2}{\sum_{i}\left(x_{i}-\bar{x}\right)^{2}\sum_{i}\left(y_{i}-\bar{y}\right)^{2}}=\frac{(\sum_{i}\left(x_{i}-\bar{x}\right)\left(r_i+\hat{y}_{i}-\bar{y}\right))^2}{\sum_{i}\left(x_{i}-\bar{x}\right)^{2}\sum_{i}\left(y_{i}-\bar{y}\right)^{2}}\\
       &=\frac{(\sum_{i}\left(x_{i}-\bar{x}\right)\left(\hat{y}_{i}-\bar{y}\right))^2}{\sum_{i}\left(x_{i}-\bar{x}\right)^{2}\sum_{i}\left(y_{i}-\bar{y}\right)^{2}}=\frac{(\sum_{i}\left(x_{i}-\bar{x}\right)\left(\hat{y}_{i}-\bar{y}\right))^2}{\sum_{i}\left(x_{i}-\bar{x}\right)^{2}\sum_{i}\left(y_{i}-\bar{y}\right)^{2}}
    \end{aligned}
    \nonumber
\end{equation}










\end{document}