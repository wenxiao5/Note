%!TEX program = xelatex
\documentclass[11pt,a4paper]{article}
\usepackage[utf8]{inputenc}
\usepackage[T1]{fontenc}
\usepackage{authblk}
\usepackage{tikz}
\usepackage{pgfplots}
\usepackage{verbatim}
\usepackage{amsfonts}
\usepackage{amsmath}
\usepackage{amsthm}
\usepackage{indentfirst}
\usepackage{amssymb}
\setlength{\parindent}{0pt}
\usetikzlibrary{shapes,snakes}
\newcommand{\argmax}{\operatornamewithlimits{argmax}}
\newcommand{\argmin}{\operatornamewithlimits{argmin}}
\DeclareMathOperator{\col}{col}
\usepackage{booktabs}
\newtheorem{theorem}{Theorem}
\newtheorem{note}{Note}
\newtheorem{definition}{Definition}
\newtheorem{proposition}{Proposition}
\newtheorem{lemma}{Lemma}
\newtheorem{example}{Example}
\newtheorem{corollary}{Corollary}
\usepackage{graphicx}
\usepackage{geometry}
\usepackage{hyperref}
\newcommand{\code}{	exttt}
\geometry{a4paper,scale=0.8}
\title{STAT 425 Final Project}
\author[*]{Wenxiao Yang}
\affil[*]{Department of Mathematics, University of Illinois at Urbana-Champaign}
\date{2021}
\begin{document}
\maketitle

\section{Introduction}
This project aims to identify the optimal operating conditions for Company XX's bubble wrap lines in order to increase production capacity. We investigate the results of a completely randomized design experiment. There are two factors: line speed (with three levels, $36,37,38 m/mm$), percent loading of additives (with three levels, 0,2,4\%). And one response: production rate (lbs/hr). The experiment was replicated three times.



















The factor effects model is as follows:
$$Y_{ijk}=\mu+\alpha_i+\beta_j+(\alpha\beta)_{ij}+\varepsilon_{ijk}$$
$Y_{ijk}$: \textit{Production rate} for \textit{Line Speed} $i$, \textit{Percent Loading of Additives} $j$, \textit{replication} $k$.\\
$\alpha_i$: effect of \textit{Line Speed} $i$ on \textit{Production rate}.\\
$\beta_j$: effect of \textit{Percent Loading of Additives} $j$ on \textit{Production rate}.\\
$(\alpha\beta)_{ij}$: interaction term.\\
The error terms satisfy the usual assumption $\varepsilon_{i j k} \sim \mathcal{N}\left(0, \sigma^{2}\right)$.\\
Sum Constraints: $\sum_{i} \alpha_{i}=0, \sum_{j} \beta_{j}=0, \sum_{i}(\alpha \beta)_{i j}=\sum_{j}(\alpha \beta)_{i j}=0$.\\

According to the estimated interaction plot, we can conclude that




\end{document}