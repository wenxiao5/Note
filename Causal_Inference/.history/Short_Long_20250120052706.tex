\chapter{Predicting Long-term Outcomes}
\section{Surrogate Index: \cite{athey2019surrogate}}
Define two samples $N_E$ (Experimental) and $N_O$ (Observational), $N=N_E+N_O$, where we use $P_i\in\{O,E\}$ as a binary indicator of whether an individual $i$ is in $N_E$ or $N_O$.

For each individual, we have
\begin{center}
    \begin{tabular}{ccc}
        \hline
            $X_i$& vector $\in\mathbb{X}$ & pre-treatment covariates for each unit\\
        \hline
            $W_i$& $\in\mathbb{W}:=\{0,1\}$ & binary treatment for each unit (unobserved if $P_i=O$)\\
        \hline
            $Y_i$& scalar $\in\mathbb{Y}$ & primary outcome (unobserved if $P_i=E$)\\
            $S_i$& vector $\in\mathbb{S}$ &surrogates (intermediate outcomes)\\
        \hline
    \end{tabular}
\end{center}
The primary outcome $Y_i$ is unobservable for individuals in $N_E$ (i.e., in experimental sample, $P_i=E$).

Following the potential outcome framework or Rubin Causal Model, each individual has two pairs of potential outcomes:
\begin{equation}
    \begin{aligned}
        Y_i\equiv Y_i(W_i)=\left\{\begin{matrix}
            Y_i(0),& W_i=0\\
            Y_i(1),& W_i=1
        \end{matrix}\right.,\ S_i\equiv S_i(W_i)=\left\{\begin{matrix}
            S_i(0),& W_i=0\\
            S_i(1),& W_i=1
        \end{matrix}\right.
    \end{aligned}
    \nonumber
\end{equation}
Overall, the units are characterized by $(Y_i(0),Y_i(1),S_i(0),S_i(1),X_i,W_i,P_i)$.
\begin{itemize}
    \item In the experimental sample, $P_i=E$, we observe $$(X_i,W_i,S_i)\in (\mathbb{X},\mathbb{W},\mathbb{S})$$
    \item In the observational sample, $P_i=O$, we observe $$(X_i,S_i,Y_i)\in (\mathbb{X},\mathbb{S},\mathbb{Y})$$
\end{itemize}
For simplicity, we analyze the data as we observe
\begin{equation}
    \begin{aligned}
        (P_i,X_i,S_i,\mathbf{1}_{P_i=E}W,\mathbf{1}_{P_i=O}Y_i)
    \end{aligned}
    \nonumber
\end{equation}
The estimand we are interested in is  the Average Treatment Effect (ATE) on the primary outcome in the population from which the experimental sample is drawn:
\begin{equation}
    \begin{aligned}
        \tau\equiv \mathbb{E}[Y_i(1)-Y_i(0)\mid P_i=E].
    \end{aligned}
    \nonumber
\end{equation}

\subsection{Critical Assumptions}
For the individuals in the experimental group, the propensity score is the conditional probability of receiving the treatment: $\rho(x)\equiv \textnormal{Pr}(W_i=1\mid X_i=x,P_i=E)$.
\begin{assumption}[Unconfounded Treatment Assignment /  Strong Ignorability]
\begin{enumerate}
    \item For individuals in the experimental group, treatment assignment is unconfounded: $$W_i\perp (Y(W_i), S(W_i))\mid X_i,P_i=E$$
    \item We have overlap in the distribution of pre-treatment variables between the treatment and control groups: $$\rho(x)\in (0,1),\ \forall x\in \mathbb{X}$$
\end{enumerate}
\end{assumption}
\textit{Surrogate} is a post-treatment variable where conditioning on it makes the outcome and the treatment independent. We can define two scalar functions of surrogates:
\begin{definition}[Surrogate Index and Surrogate Score]
    \begin{enumerate}
        \item The \textit{surrogate index} is the conditional expectation of the primary outcome given the surrogate outcomes and the pre-treatment variables, conditional on the sample: $$\mu(s,x,p)\equiv \mathbb{E}[Y_i\mid S_i=s,X_i=x,P_i=p]$$
        \item The \textit{surrogate score} is the conditional probability of having received the treatment given the value for the surrogate outcomes and the covariates in the experimental sample: $$\rho(s,x)\equiv \textnormal{Pr}(W_i=1\mid S_i=s,X_i=x,P_i=E)$$
    \end{enumerate}
\end{definition}
\begin{assumption}[Surrogacy]
    The treatment is independent of the primary outcomes conditional on the surrogates on the experimental group: $$W_i\perp Y(W_i)\mid S(W_i), X_i,P_i=E$$
\end{assumption}
\begin{definition}[Surrogacy]
    $S_i$ is a surrogate if and only if
    \begin{equation}
        \begin{aligned}
        \end{aligned}
        \nonumber
    \end{equation}
\end{definition}