\documentclass[11pt]{elegantbook}
\definecolor{structurecolor}{RGB}{40,58,129}
\linespread{1.6}
\setlength{\footskip}{20pt}
\setlength{\parindent}{0pt}
\newcommand{\argmax}{\operatornamewithlimits{argmax}}
\newcommand{\argmin}{\operatornamewithlimits{argmin}}
\elegantnewtheorem{proof}{Proof}{}{Proof}
\elegantnewtheorem{claim}{Claim}{prostyle}{Claim}
\DeclareMathOperator{\col}{col}
\title{\textbf{Mathematical Tools for Economists}}
\author{Wenxiao Yang}
\institute{Haas School of Business, University of California Berkeley}
\date{2023}
\setcounter{tocdepth}{2}
\cover{cover.png}
\extrainfo{All models are wrong, but some are useful.}

% modify the color in the middle of titlepage
\definecolor{customcolor}{RGB}{9,119,119}
\colorlet{coverlinecolor}{customcolor}
\usepackage{cprotect}

\addbibresource[location=local]{reference.bib} % bib

\begin{document}

\maketitle
\frontmatter
\tableofcontents
\mainmatter


\chapter{Basic Definitions}
\section{Main Methods of Proof}
\subsection{Proof by Induction}
\subsection{Proof by Deduction}
\subsection{Proof by Contradiction}
\subsection{Proof by Contraposition}
\begin{enumerate}[$\circ$]
    \item $\lnot P$ ("not $P$") means "$P$ is false".
    \item $P \wedge Q$ ("$P$ and $Q$") means “$P$ is true and $Q$ is true.”
    \item $P \vee Q$ ("$P$ or $Q$") means “$P$ is true or $Q$ is true (or possibly both).”
    \item $\lnot P \wedge Q$ means $(\lnot P)\wedge Q$; $\lnot P \vee Q$ means $(\lnot P)\vee Q$.
    \item $P \Rightarrow Q$ ("$P$ implies $Q$") means “whenever $P$ is satisfied, $Q$ is also satisfied.”
\end{enumerate}
\textbf{Statement:} Formally, $P \Rightarrow Q$ is equivalent to $\lnot P \vee Q$.

\begin{definition}[Contrapositive]
\normalfont
The \textit{contrapositive} of the statement $P \Rightarrow Q$ is the statement $\lnot Q \Rightarrow \lnot P$.
\end{definition}

\begin{theorem}[Prove Contrapositive Insead]
\normalfont
$P \Rightarrow Q$ is true if and only if $\lnot Q \Rightarrow \lnot P$ is true.
\end{theorem}









\end{document}